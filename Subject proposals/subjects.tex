\documentclass[a4paper]{article}
\usepackage{graphicx} % Required for inserting images
\usepackage[a4paper, margin=1.8cm]{geometry}

\title{HEIG-VD --- EAL 2024 --- Group \textit{"Groupe"}}

% it italics
% tt truetype
% bf boldface

\author{
    Van Hove, Timothée (ISCL) \textit{Spokesperson}\\\texttt{timothee.vanhove@heig-vd.ch}\\
    Mangold, Aubry (ISCS) \textit{Deputy spokesperson}\\\texttt{aubry.mangold@heig-vd.ch}\\
    Strefeler, Michael (ISCD)\\\texttt{michael.strefeler@heig-vd.ch}\\
    Gillet, Paul (ISCD)\\\texttt{paul.gillet@heig-vd.ch}\\
    Auberson, Kevin (ISCR)\\\texttt{kevin.auberson@heig-vd.ch}\\
}

\date{\today}

\begin{document}

\maketitle

\section{Topic proposals}

In order of preference.

\subsection{Genomics service}

A service that collects users genetic information, to offer personalized genetic analyses to inform users about their health, their predisposition to certain diseases and their response to medication.

Legal challenges involve compliance with data protection and privacy that set strict guidelines on the collection, use, and sharing of genetic data. The service must also comply with healthcare regulations, ensuring that any health-related advice or analysis provided meets the standards set by medical authorities.

Ethical challenges lies in the risk of misuse of genetic information by third parties, which could lead to discrimination in insurance, or social stigmatization. Moreover, the service has the responsibility to ensure that genetic data is interpreted with care and communicated to users in a way that is accurate and supportive, considering the psychological implications of discovering predisposition to diseases or specific medical conditions.

\subsection{Spatial computing for medical imaging and diagnostics}

A spacial computing device used to gather and manipulate data from patients. The technology allows medical professionals to visualize patient data, plan surgeries and guide medical procedures.

Legal challenges are concerned with patient data protection and regulatory compliance for medical devices. Dealing with such sensitive data can be complicated and patients need to give their consent and be sure that their data won't be used with malicious intent. The device must meet regulatory standards for medical devices that include requirements for safety, efficacy, and reliability.

Ethical challenges lies in patient consent and the accuracy and reliability of the technology. Patients must understand how their data will be used. There is also the need to ensure that the device provides reliable data to inform medical decisions, given the potential for harm if the technology fails. Moreover, this technology raises questions about equitable access to healthcare,

\subsection{Generative AI for teaching applications}

A service that uses generative AI to produce personalized educational content for students that provides specialized teachers generated through ML and customizes feedback to students based on performance and difficulties. 

Legal challenges are data privacy and intellectual property rights. The collection and analysis of student data to personalize content and feedback needs to adhere to data protection laws such as GDPR. The use of AI to generate educational material raises questions about copyright and ownership of the content, especially when the AI is trained upon existing copyrighted resources to create new content.

Ethical challenges lies in ensuring fairness and bias mitigation in the AI algorithms that personalize educational content and generate virtual teachers. Moreover, the reliance on AI-generated teachers and content might raise concerns about the quality of human interaction, the potential loss of teacher-student relationships, and the impact on students social and emotional development
\end{document}
